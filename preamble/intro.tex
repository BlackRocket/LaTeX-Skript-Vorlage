\begingroup
	%Titelseite setzen
	\begin{titlepage}
		\begin{center}
			\makeatletter
				\includegraphics[width=3cm]{\@logo}\\[6cm]
				\textbf{{\LARGE Script}\\[0.5cm]}
			\makeatother
			{\bfseries
				\makeatletter
					\Huge\@title
				\makeatother
			}\\[3cm]
			\makeatletter
				\noindent\textbf{{\large \@author}}\\[16pt]
				{\normalsize \@ort, \today}
			\makeatother
		\end{center}
		\vspace*{\fill} 
		\begin{minipage}[b]{8cm}
			\begin{flushleft}
				\makeatletter
					\@lizenz
				\makeatother
			\end{flushleft}
		\end{minipage}
		\hfill
		\begin{minipage}[b]{8cm} 
			\begin{flushright}
				\makeatletter
					\@kontakt
				\makeatother
			\end{flushright}
		\end{minipage}
	\end{titlepage}
\endgroup
%Römische Ziffern (groß) zur Nummerierung des Vorspanns
\pagenumbering{Roman}
\clearpage\setcounter{page}{1}
% Vorwort oder der gleichen

\chapter*{Vorwort}


\clearpage

% Inhaltsverzeichnis (Gliederung) mit einer tiefe von 3 erstellen
\setcounter{tocdepth}{3}
\tableofcontents
\clearpage

% Literaturverzeichnis jurlit.bib
\nocite{*} % alle Eintrge in der Bib anzeigen
\bibliography{script}
\bibliographystyle{jurabib}
\clearpage