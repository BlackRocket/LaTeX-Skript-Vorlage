% ========
% Jura

% stellt die Markierungen in einer enumerate Umgebung auf diejenigen um, die in der Rechtswissenschaft üblich sind: I, 2, c) dd). (via jurabase)
\juraenum

% Das Inhaltsverzeichnis hei"st Gliederung
\renewcommand{\contentsname}{Gliederung}

% Entnommen jur-grie.sty von Peter Felix Schuster (http://www.peterfelixschuster.de)
%Definition \greekalpha f"ur (kursive) griechische Kleinbuchstaben (aus dem
%mathematischen Modus, daher \ensuremath{}).
%Aufruf mit \greekalpha{<z"ahler>}, wobei <z"ahler> eine Zahl von 1-24 sein muss. Eignet sich
%also eher f"ur Aufz"ahlungen, da deren Z"ahler immer einen Zahlenwert "ubergibt.
%Werden von einem Buchstaben mehrere Darstellungsweisen angeboten (\phi,
%\varphi), habe ich die gew"ahlt die im Duden (zuerst) stand.
%Der Z"ahler muss mit \value (etwa \value{enumiv}) "ubergeben werden, sonst
%gibt's eine Fehlermeldung (no number given oder so). 
\newcommand{\greekalpha}[1]{\ensuremath{\ifcase\value{#1}\or \alpha\or \beta\or%
    \gamma\or \delta\or \varepsilon\or \zeta\or \eta\or \theta\or \iota\or%
    \kappa\or \lambda\or \mu\or \nu\or \xi\or o\or \pi\or \varrho\or \sigma\or%
    \tau\or \upsilon\or \varphi\or \chi\or \psi\or  \omega\fi}}

%In Aufz"ahlungen eher selten ben"otigt werden griechische Gro"sbuchstaben,
%aber \Greekalpha (gro"ses G beachten!) stellt sie trotzdem bereit. Rest wie
%oben, au"ser, dass sie nicht kursiv sind und weitgehend mit den lateinischen
%Gro"sbuchstaben identisch
\newcommand{\Greekalpha}[1]{\ensuremath{\ifcase\value{#1}\or \mbox{A}\or \mbox{B}\or%
    \Gamma\or \Delta\or \mbox{E}\or \mbox{Z}\or \mbox{H}\or \Theta\or \mbox{I}\or%
    \mbox{K}\or \Lambda\or \mbox{M}\or \mbox{N}\or \Xi\or \mbox{O}\or \Pi\or \mbox{P}\or%
    \Sigma\or \mbox{T}\or \Upsilon\or \Phi\or \mbox{X}\or \Psi\or \Omega\fi}}

% Entnommen jur-rdnr.sty
% von Peter Felix Schuster (http://www.peterfelixschuster.de)
\makeatletter
%Z"ahler f"ur F"alle und Abwandlungen definieren und setzen
\newcommand{\@rdnrname}{Rn.\,}%Makro f"ur die Bezeichnung f"ur F"alle, macht es flexibel zu "andern
\newcounter{randnr}%Z"ahler randnr (f"ur Randnummer)
\newcounter{zwrandnr}[randnr]%Z"ahler zwrandnr (f"ur Zwischen-Randnummer), eine Erh"ohung von randnr setzt ihn zur"uck
\setcounter{randnr}{0}%Anfangs auf 0 setzen
\setcounter{zwrandnr}{0}%Anfangs auf 0 setzen
\newcommand{\newrandnummer}{\refstepcounter{randnr}}%Makro, das den Z"ahler erh"oht
\newcommand{\newzwischenrandnummer}{\refstepcounter{zwrandnr}}%Makro, das den Z"ahler erh"oht
\renewcommand{\therandnr}{\arabic{randnr}}%In arabischen Ziffern nummerieren.
\renewcommand{\thezwrandnr}{\therandnr\,\alph{zwrandnr}}%Kleinbuchstaben nach ''normaler'' Randnummer.
\renewcommand{\p@randnr}{\@rdnrname}%Bei Verweisen Rdnr. vor Zahl ausgegeben
\renewcommand{\p@zwrandnr}{\@rdnrname}%Bei Verweisen Rdnr. vor Zahl\,Buchstabe ausgegeben

% marginpar Config
\setlength{\marginparwidth}{15mm}
\marginparsep5mm
\normalmarginpar

%Z"ahler "uber Makro erh"ohen und am Rand als Randnummer ausgeben
%Kann einfach am Anfang oder vor dem zu beziffernden Absatz, vor allem aber vor eventuellen \label-Befehlen als
%\randnummer{} eingef"ugt werden. Allerdings muss es nach Abschnittsbefehlen (wie \section) eingef"ugt werden.
\newcommand{\randnummer}{\newrandnummer\marginpar{\vspace{-30pt}{\strut\\\centering\textbf{\therandnr}}}}

%Dieses Makro ist f"ur Zwischenrandnummern gedacht, also nachtr"aglich eingef"ugten, etwa 36a, 36b etc. Die Benutzung
%ist ansonsten die gleiche.
\newcommand{\zwischenrandnummer}{\newzwischenrandnummer\marginpar{\strut\\\centering\textbf{\thezwrandnr}}}

%Was leider noch nicht funktioniert, ist, dass auch ein Inhaltsverzeichnis der Randnummern angelegt wird.

\newcommand{\Punkte}[1]{\marginpar{\strut\\\centering\textbf{#1}}}
\makeatother

% Entnommen jur-aufz.sty
% von Peter Felix Schuster (http://www.peterfelixschuster.de)
\makeatletter
%Juristische Nummerierung f"ur Aufz"ahlungen 1. a) aa) (1)
%Geht davon aus, dass die h"ohere Ebene, also I. mit \subsection erzeugt
%wird. 
%\theenumi setzt das eigentliche Z"ahlerformat. Das ist vor allem f"ur Verweise
%wichtig, s.u. 
%\labelenumi setzt die Darstellungsform des Z"ahlers. Hier identisch. Damit
%\LaTeX\ nicht selbst was dranh"angt 
%erste Ebene: 1. (arabische Ziffern Punkt), Standardwert
\renewcommand{\theenumi}{\arabic{enumi}}
\renewcommand{\labelenumi}{\theenumi.}
%zweite Ebene: a) (lateinische Kleinbuchstaben Klammer)
\renewcommand{\theenumii}{\alph{enumii}}
\renewcommand{\labelenumii}{\theenumii)}
%dritte Ebene: aa) (doppelte lateinische Kleinbuchstaben Klammer)
\renewcommand{\theenumiii}{\alph{enumiii}\alph{enumiii}}
\renewcommand{\labelenumiii}{\theenumiii)}
%vierte Ebene: alpha) (griechische Kleinbuchstaben Klammer).
\renewcommand{\theenumiv}{\greekalpha{enumiv}}
\renewcommand{\labelenumiv}{\theenumiv)}

% Um Verweisungen anzupassen das Makro \oberebene{#1} 
% Normalerweise wird bei Verweisen die "ubergeordnete Gliederungsebene (section+subsection) ausgegeben.
% Will man das nicht (etwa in einem Schema, in einem Anspruch), dann kann man es anpassen, etwa mit
%     \oberebene{\thenapruchnr}
% Mit der Sternchenvariante (ohne Parameter) stellt man wieder zur"uck:
%     \oberebene*
\newcommand{\@oberebene}{\thesection\,\thesubsection}
\newcommand{\oberebene}{\@ifstar{\@@oberebenen}{\@@oberebene}}
\newcommand{\@@oberebene}[1]{\renewcommand{\@oberebene}{#1}}
\newcommand{\@@oberebenen}{\renewcommand{\@oberebene}{\thesection\,\thesubsection}}

%Korrektur f"ur Referenzen auf Aufz"ahlungselemente. Hinweise zu
%\p@... s.o. 
%Achtung! Hier werden section und subsection immer mitzitiert. Ist die
%Aufz"ahlung etwa im Unterabschnitt A. I. und soll auf Punkt 1. a) der
%Aufz"ahlung verwiesen werden, so lautet die Verweisung ''A. I. 1. a)''! Es
%k"onnen Fehler entstehen, wenn die Aufz"ahlung direkt in einem Abschnitt
%(\section) steht, ohne dass ein Unterabschnitt (\subsection) begonnen wurde!
%Wer das dennoch vorhat, muss anpassen!
% \renewcommand{\p@enumi}{\thesection\,\thesubsection\,}
% \renewcommand{\p@enumii}{\thesection\,\thesubsection\,\theenumi.\,}
% \renewcommand{\p@enumiii}{\thesection\,\thesubsection\,\theenumi.\,\theenumii)\,}
% \renewcommand{\p@enumiv}{\thesection\,\thesubsection\,\theenumi.\,\theenumii)\,\theenumiii)\,}
\renewcommand{\p@enumi}{\@oberebene\,}
\renewcommand{\p@enumii}{\@oberebene\,\theenumi.\,}
\renewcommand{\p@enumiii}{\@oberebene\,\theenumi.\,\theenumii)\,}
\renewcommand{\p@enumiv}{\@oberebene\,\theenumi.\,\theenumii)\,\theenumiii)\,}


%Formatierung f"ur nicht nummerierte Listen - mathematische Symbole
\renewcommand{\labelitemi}{$\m@th\bullet$} %erste Ebene: Fetter schwarzer Punkt (Standard)
\renewcommand{\labelitemii}{--}%zweite Ebene: Gedankenstrich
%Vorher: {\Rightarrow} Pfeil nach rechts, $\m@th\diamond$ Karo
\renewcommand{\labelitemiii}{$\m@th\circ$}%dritte Ebene: Leerer Kreis
%Vorher: {\Pointinghand} Hand 
\renewcommand{\labelitemiv}{$\m@th\rightarrow$}%vierte Ebene: Rechts-Pfeil
%Vorher: {\Crossedbox}Ankreuz-K"astchen
\makeatother


%Abk"urzungen - vornehmlich f"ur schmale Zwischenr"aume (\,) n"utzlich
%ben"otigt f"ur einige Befehle jurabib. Befehl nachschlagen, mit dem "uberpr"uft werden kann, ob geladen!
\providecommand{\Abs}[1]{Abs.\,#1}
\providecommand{\aF}{a.\,F.\xspace} %\aF => a. F.
\providecommand{\andM}{\textbf{a.\,M.}\xspace} %\andM => a. M. in Fettdruck
\providecommand{\euro}[1]{#1\,\EUR} %\euro{x} => x (Eurosymbol)
\providecommand{\folg}{\,f.\xspace} %f. nach kurzem Leerraum
\providecommand{\ffolg}{\,ff.\xspace} %ff. nach kurzem Leerraum
\providecommand{\hL}{h.\,L.\xspace} %\hL => h.L.
\providecommand{\hM}{h.\,M.\xspace} %\hM => h. M.
\providecommand{\idR}{i.\,d.\,R.\xspace} %\idR => i. d. R.
\providecommand{\iHv}{i.\,H.\,v.\xspace} %\iHv => i. H. v.
\providecommand{\iSd}{i.\,S.\,d.\xspace} %\iSd => i. S. d.
\providecommand{\iSv}{i.\,S.\,v.\xspace} %\iSv => i. S. v.
\providecommand{\iVm}{i.\,V.\,m.\xspace} %\iVm => i. V. m.
\providecommand{\ja}{\ding{'63}\xspace}%\ensuremath{\bigoplus}%(+) 
\providecommand{\mwN}{m.\,w.\,N.\xspace} %\mwN => m. w. N.
\providecommand{\LWL}{\bfseries\underline{\emph{LWL}}}%steht f"ur Literaturwunschliste, also f"ur Blindzitate. Alles folgende wird fett!
\providecommand{\nein}{\ding{'67}\xspace}%\ensuremath{\ominus}%(-)
\providecommand{\nF}{n.\,F.\xspace} %\nF => n. F.
\providecommand{\Nr}[1]{Nr.\,#1}
\providecommand{\Nro}[1]{Nr.\,#1}
\providecommand{\pg}[1]{\S\,#1} %\pg{x} => (Paragraf) x
\providecommand{\Pg}[1]{\SSS\,#1} %\Pg{x} => (Paragrafen) x 
\providecommand{\pgAbs}[2]{\S\,#1 Abs.\,#2} %\pgAbs{x}{y} => (Paragraf) x Abs. y
\providecommand{\pgAbsS}[3]{\S\,#1 Abs.\,#2 S.\,#3}
\providecommand{\pgRn}[2]{\S\,#1 Rn.\,#2} %\pg{x}{y} => (Paragraf) x Rn. y
\providecommand{\pgS}[2]{\S\,#1 S.\,#2}
\providecommand{\Rn}[1]{Rn.\,#1}
\providecommand{\Satz}[1]{S.\,#1}
\providecommand{\Seite}[1]{S.\,#1}
\providecommand{\zBsp}{z.\,B.}
\providecommand{\fremdwort}[2]{\foreignlanguage{#1}{\itshape{}#2}}%\fremdwort{#1=Sprache}{#2=Text}

%\providecommand{\altzahl}[1]{{\oldstyle #1}}
%\DeclareTextSymbol{\textNr}{TS1}{'233}
