%%% Packages for LaTeX - programming
%
% Define commands that don't eat spaces.
\usepackage{xspace}
% IfThenElse
\usepackage{ifthen}
%%% Doc: ftp://tug.ctan.org/pub/tex-archive/macros/latex/contrib/oberdiek/ifpdf.sty
% command for testing for pdf-creation
\usepackage{ifpdf} %\ifpdf \else \fi

%%% Internal Commands: ----------------------------------------------
\makeatletter
%
\providecommand{\IfPackageLoaded}[2]{\@ifpackageloaded{#1}{#2}{}}
\providecommand{\IfPackageNotLoaded}[2]{\@ifpackageloaded{#1}{}{#2}}
\providecommand{\IfElsePackageLoaded}[3]{\@ifpackageloaded{#1}{#2}{#3}}
%
\newboolean{chapteravailable}%
\setboolean{chapteravailable}{false}%

\ifcsname chapter\endcsname
  \setboolean{chapteravailable}{true}%
\else
  \setboolean{chapteravailable}{false}%
\fi


\providecommand{\IfChapterDefined}[1]{\ifthenelse{\boolean{chapteravailable}}{#1}{}}%
\providecommand{\IfElseChapterDefined}[2]{\ifthenelse{\boolean{chapteravailable}}{#1}{#2}}%

\providecommand{\IfDefined}[2]{%
\ifcsname #1\endcsname
   #2 %
\else
     % do nothing
\fi
}

\providecommand{\IfElseDefined}[3]{%
\ifcsname #1\endcsname
   #2 %
\else
   #3 %
\fi
}

\providecommand{\IfElseUnDefined}[3]{%
\ifcsname #1\endcsname
   #3 %
\else
   #2 %
\fi
}


%
% Check for 'draft' mode - commands.
\newcommand{\IfNotDraft}[1]{\ifx\@draft\@undefined #1 \fi}
\newcommand{\IfNotDraftElse}[2]{\ifx\@draft\@undefined #1 \else #2 \fi}
\newcommand{\IfDraft}[1]{\ifx\@draft\@undefined \else #1 \fi}
%

% Definde frontmatter, mainmatter and backmatter if not defined
\@ifundefined{frontmatter}{%
   \newcommand{\frontmatter}{%
      %In Roemischen Buchstaben nummerieren (i, ii, iii)
      \pagenumbering{roman}
   }
}{}
\@ifundefined{mainmatter}{%
   % scrpage2 benoetigt den folgenden switch
   % wenn \mainmatter definiert ist.
   \newif\if@mainmatter\@mainmattertrue
   \newcommand{\mainmatter}{%
      % -- Seitennummerierung auf Arabische Zahlen zuruecksetzen (1,2,3)
      \pagenumbering{arabic}%
      \setcounter{page}{1}%
   }
}{}
\@ifundefined{backmatter}{%
   \newcommand{\backmatter}{
      %In Roemischen Buchstaben nummerieren (i, ii, iii)
      \pagenumbering{roman}
   }
}{}

% Pakete speichern die spaeter geladen werden sollen
\newcommand{\LoadPackagesNow}{}
\newcommand{\LoadPackageLater}[1]{%
   \g@addto@macro{\LoadPackagesNow}{%
      \usepackage{#1}%
   }%
}



\makeatother
%%% ---------------------------------------------------------------- % Quelle http://www.matthiaspospiech.de/latex/vorlagen/allgemein/preambel/
\sloppy
%----------------------------------
% laden der Verschiedenen Pakete
%----------------------------------
% erweitertes Farbpaket
\usepackage{xcolor}

% Grafikpaket
\usepackage{graphicx}

%========
% Schrift & Sprache
% Schriftsatz utf8 und Deutsch
\usepackage[utf8]{inputenc} % Schriftsatz utf8 verwenden 
\usepackage[T1]{fontenc} % T1-Fonts
\usepackage[ngerman]{babel} % Deutsche Sonderzeichen und Silbentrennung (neue Rechtschreibung)

% Schrift
\usepackage{mathptmx} % Times Roman als Standardschrift
\usepackage[scaled=.92]{helvet} % Arial für die Überschriften
\usepackage{courier} % PostScript Schriftarten (Courier)
\usepackage{marvosym} % diverse Symbole
\usepackage{wasysym}
\usepackage{eurosym} % Euro Symbole


% franz. Anführungszeichen verwenden
\usepackage[babel,german=guillemets]{csquotes}

% für Greek im Text
\usepackage{textgreek}
\usepackage{eulervm}
	
% behebt Makrogrenz-Problem
\usepackage{xspace}

% erlaubt es Text zu drehen
\usepackage{rotating}
%========
% Tabellen
% bessere Tabellen
\usepackage{tabularx}

%Fussnoten innerhalb von Tabellen
\usepackage[para]{threeparttable}

% erlaubt Farbige Zellen
\usepackage{colortbl}

\usepackage{booktabs}

%========
% andere Pakete

% Jurabib für die Bibliographie
\usepackage{jurabib}

% Paket zur Texthervorhebung
\usepackage{soul}

% für hervor gehobene Boxen
\usepackage{framed} 

% Anpassbare Enumerates/Itemizes
\usepackage{enumitem}

% zur Darstellung von Definitionen
\usepackage[amsthm,thmmarks]{ntheorem}

% für Zeilenabstand
\usepackage{setspace} 

% für Randnotizen (von den Randnummern verwendet)
\usepackage{marginnote}

% ermöglicht das anpassen der Titel
\usepackage{titlesec}


%----------------------------------
% Pakete mit Config
%----------------------------------

% normaler Seitenrand
\usepackage[%
	a4paper, 
	left=25mm, 
	right=25mm, 
	top=3cm, 
	bottom=3cm
]{geometry}

\usepackage[
   % Farben fuer die Links
   colorlinks=true,         % Links erhalten Farben statt Kaeten
   urlcolor=pdfurlcolor,    % \href{...}{...} external (URL)
   filecolor=pdffilecolor,  % \href{...} local file
   linkcolor=pdflinkcolor,  % \ref{...} and \pageref{...}
   % Links
   raiselinks=true,			 % calculate real height of the link
   backref=false,            % Backlinks im Literaturverzeichnis (section, slide, page, none)
   pagebackref=false,         % Backlinks im Literaturverzeichnis mit Seitenangabe
   verbose,
   hyperindex=true,         % backlinkex index
   linktocpage=true,        % Inhaltsverzeichnis verlinkt Seiten
   hyperfootnotes=true,     % Keine Links auf Fussnoten
   % Bookmarks
   bookmarks=true,          % Erzeugung von Bookmarks fuer PDF-Viewer
   bookmarksopenlevel=1,    % Gliederungstiefe der Bookmarks
   %bookmarksopen=true,      % Expandierte Untermenues in Bookmarks
   bookmarksnumbered=true,  % Nummerierung der Bookmarks
   bookmarkstype=toc,       % Art der Verzeichnisses
   % Anchors
   plainpages=false,        % Anchors even on plain pages ?
   pageanchor=true,         % Pages are linkable
   % PDF Informationen
   pdftitle={},             % Titel
   pdfauthor={},            % Autor
   pdfcreator={LaTeX, hyperref, KOMA-Script}, % Ersteller
   %pdfproducer={pdfeTeX 1.10b-2.1} %Produzent
   pdfstartview=FitH,       % Dokument wird Fit Width geaefnet
   pdfpagemode=UseOutlines, % Bookmarks im Viewer anzeigen
   pdfpagelabels=true,      % set PDF page labels
]{hyperref}

% URL Paket (darf erst nach hyperref geladen werden
\usepackage{url} 


%----------------------------------
% Config
%----------------------------------

% ========
% Bibliographie und Fußnoten

% Konfiguriert das jurabib-Paket. Kommentare aus der Dokumentation
\jurabibsetup{%
	authorformat=italic, 		% Autor kursiv
	titleformat={commasep,all}, % Komma zwischen Autor/Bearbeiter und Titel im Zitat;
	annotatorformat=italic, 	% Bearbeiter kursiv
	annotatorlastsep=divis, 	% Bearbeiter nach Bindestrich
	commabeforerest, 			% Komma nach Verfasser (vor dem Rest)
	crossref={long,dynamic}, 	% Lange Querverweise (auf Festschriften etwa)
	howcited=compare, 			%"zitiert als...", wenn shorttitle anders als title
	pages={always,test}, 		% zitierten Seitenbereich immer ausgeben (always),
	bibformat={tabular,ibidem}, % Litverz. tabellarisch, mit der-/dieselbe
	lookforgender, 				% Auf das gender-Feld achten, um ders./dies. Zitate zu ermglichen
	superscriptedition=switch, 	% Hochgestellte Auflage, wenn ssedition=1 in .bib
	dotafter=bibentry, 			% Punkt nach jedem Eintrag im Lit.verzeichnis
}

\citetitlefortype{article,periodical,incollection} % Diese immer mit Titel zitieren
\formatpages[~]{article}{(}{)} % Zeitschriften als JZ 2001, 1057, (S.) %[, ]
\formatpages[~]{incollection}{(}{)} % Sammbelbandbeiträge als FS xy, 1057, (S.) %[, ]

%Bei Festschriften und Zeitschriftenartikeln: "`in"' vor Titel der Sammlung
\renewcommand{\bibjtsep}{In: } 
\renewcommand{\bibbtsep}{In: } 

%Bei Periodika (AcP et.al.) die Jahreszahl in runde (statt eckige) Klammern setzen.
\renewcommand*{\bibpldelim}{(}
\renewcommand*{\bibprdelim}{)}

%Linke Spalte des Lit.verz. soll ein Drittel der ges. Textbreite einnehmen
\renewcommand*{\bibleftcolumn}{\textwidth/3}

%Nicht Punkt, sondern Komma nach Auflage
\DeclareRobustCommand{\jbaensep}{,}

%Bei Artikeln: Heft-Nummer in Klammern hinter dem Erscheinungsjahr, etwa 2002(7). (aus jurabib-Gruppe)
\DeclareRobustCommand{\artnumberformat}[1]{(#1)}

%Kein Komma hinter Zeitschriftenname (aus: jurabib-Gruppe #661)
\AddTo\bibsgerman{\def\ajtsep{}}

% Keine hochgestellten Ziffern in der Fussnote (KOMA-Script-spezifisch):
\deffootnote{1.5em}{1em}{\makebox[1.5em][l]{\thefootnotemark}}
\addtolength{\skip\footins}{\baselineskip} % Abstand Text <-> Fussnote
 
\setlength{\dimen\footins}{10\baselineskip} % Beschraenkt den Platz von Fussnoten auf 10 Zeilen
 
\interfootnotelinepenalty=10000 % Verhindert das Fortsetzen von Fussnoten auf der gegenüberligenden Seite
                                
%fuer URL (nur wenn url geladen ist)
\IfDefined{urlstyle}{
	\urlstyle{tt} % URLs in Typewriter
}

% ========
% Farben 
% Farbe der Ueberschriften
\definecolor{sectioncolor}{RGB}{0, 0, 0}    % Schwarz
%
% Farbe des Textes
\definecolor{textcolor}{RGB}{0, 0, 0}        % Schwarz
\definecolor{darkgreen}{RGB}{0,100,0}		 % Dunkelgrün
\definecolor{darkred}{RGB}{139,0,0}			 % Dunkelrot
\definecolor{grey97}{RGB}{247,247,247}		 % grey97
%
% Farbe fuer grau hinterlegte Boxen (fuer Paket framed.sty)
\definecolor{shadecolor}{gray}{0.90}

% Farben fuer die Links im PDF
\definecolor{pdfurlcolor}{rgb}{0.6,0,0}
\definecolor{pdffilecolor}{rgb}{0,0,0}
\definecolor{pdflinkcolor}{rgb}{0,0,0}
 
% Farben fuer Listings
%\colorlet{stringcolor}{green!40!black!100}
\colorlet{commencolor}{blue!0!black!100}

% ========
% Layout

% Absätze nicht einrücken
\setlength{\parindent}{0pt}


% keine automatische Nummerierung der Überschriften
\setcounter{secnumdepth}{-1} 

% Abstand zwischen chapter und Kopf um 1cm verkleinern
\renewcommand*\chapterheadstartvskip{\vspace*{-1cm}} 

% \part angepasst
\titleformat{\part}[display]{\normalfont\bfseries\LARGE\filcenter}{}{.5em}{}

% für lange Zitate
\newenvironment{bquote}% 
   {\begin{quotation}\small}% 
   {\end{quotation}}% 
\SetBlockEnvironment{bquote}

% für kurze Zitate
\newenvironment{squote}% 
   {\begin{quote}\small}% 
   {\end{quote}}% 
\SetBlockEnvironment{squote}

% \chapter nicht auf neuer Seite beginnen
%\makeatletter
%\renewcommand{\chapter}{\par
%  \thispagestyle{\chapterpagestyle}%
%  \global\@topnum\z@
%  \@afterindentfalse
%  \secdef\@chapter\@schapter
%}
%\makeatother

% neuer Tabellenspaltentyp C (X zentriert)
\newcolumntype{C}{>{\centering\arraybackslash}X}

% ========
% Definitionen und andere
\theoremstyle{nonumberplain}
\theoremseparator{:}
\theoremsymbol{}
\newtheorem{defi}{Definition}

% ========
% BOXEN

% Rahmen: schwarz; Füllung: weiß; Aufruf: wbox
\newenvironment{wbox}{%
  \def\FrameCommand{\fboxrule 0.2mm \fcolorbox{black}{white}}%
  \MakeFramed {\advance\hsize-\width \FrameRestore}}%
 {\endMakeFramed}
 
% Rahmen: weiß; Füllung: weiß; Aufruf: fuellbox
 \newenvironment{fuellbox}{%
  \def\FrameCommand{\fboxrule 0.2mm \fcolorbox{white}{white}}%
  \MakeFramed {\advance\hsize-\width \FrameRestore}}%
 {\endMakeFramed}

% Rahmen: schwarz; Füllung: helles grau; Aufruf: defaultbox
\newenvironment{defaultbox}{%
  \def\FrameCommand{\fboxrule 0.5mm \fcolorbox{black}{lightgray}}%
  \MakeFramed {\advance\hsize-\width \FrameRestore}}%
 {\endMakeFramed}
 
% Rahmen: schwarz; Füllung: helles grau; Aufruf: defaultbox
\newenvironment{darkgreen}{%
  \def\FrameCommand{\fboxrule 0.5mm \fcolorbox{darkgreen}{white}}%
  \MakeFramed {\advance\hsize-\width \FrameRestore}}%
 {\endMakeFramed}
 
% Rahmen: rot; Füllung: gelb; Aufruf: wichtig
\newenvironment{wichtig}{%
  \def\FrameCommand{\fboxrule 0.5mm \fcolorbox{red}{yellow}}%
  \MakeFramed {\advance\hsize-\width \FrameRestore}}%
 {\endMakeFramed}

% ========
% Jura spezial

% \pg{x} => (Paragraf) x
\newcommand{\pg}[1]{\S\,#1} 

% \Pg{x} => (Paragrafen) x 
\newcommand{\Pg}[1]{\SSS\,#1}

% =========
% Abkuerzungen Quelle jurabase.sty

\newcommand*{\abkwarning}{
	Keine Abkuerzungen am Satzanfang!}

\newcommand*{\hM}{% Abkuerzung fuer herrschende Ansicht
h.\,M.\xspace}

\newcommand*{\aF}{% Abkuerzung fuer alte Fassung
a.\,F.\xspace}

\newcommand*{\aA}{% Abkuerzung fuer andere Ansicht
a.\,A.\xspace}

\newcommand*{\AnA}{% Abkuerzung fuer andere Ansicht - die wird zugelassen, weil ueblich!
A.\,A.\xspace}

\newcommand*{\aM}{% Abkuerzung fuer andere(r) Meinung 
a.\,M.\xspace}

\newcommand*{\aaO}{% Abkuerzung fuer am angegebenen Ort
a.\,a.\,O.\xspace}

\newcommand*{\aE}{% Abkuerzung fuer am Ende
a.\,E.\xspace}

\newcommand*{\iVm}{% Abkuerzung fuer in Verbindung mit
i.\,V.\,m.\xspace}

\newcommand*{\iHv}{% Abkuerzung fuer in Hoehe von
i.\,H.\,v.\xspace}

\newcommand*{\zB}{% Abkuerzung fuer z.B.
z.\,B.\xspace}

\newcommand*{\ZB}{%
\abkwarning% 
Zum Beispiel\xspace}

\newcommand*{\iS}{% Abkuerzung fuer i.S.
i.\,S.\xspace}

\newcommand*{\iUe}{% Abkuerzung fuer im Uebrigen (neue Rechtschreibung)
i.\,\"U.\xspace}%

\newcommand*{\iA}{% Abkuerzung fuer im Allgemeinen neue Rechtschreibung laut Duden
i.\,Allg.\xspace}

\newcommand*{\iE}{% Abkuerzung fuer im Ergebnis
i.\,Erg.\xspace}

\newcommand*{\IE}{% Abkuerzung fuer Im Ergebnis
\abkwarning%
Im Ergebnis\xspace}

\newcommand*{\iR}{% Abkuerzung fuer i.R.
i.\,R.\xspace}

\newcommand*{\sa}{% Abkuerzung fuer siehe auch
s.\,a.\xspace}

\newcommand*{\Sa}{% Abkuerzung fuer Siehe auch
\abkwarning%
Siehe auch\xspace}

\newcommand*{\So}{% Abkuerzung fuer Siehe oben
\abkwarning%
Siehe oben\xspace}

\newcommand*{\su}{% Abkuerzung fuer siehe unten
s.\,u.\xspace}

\newcommand*{\Su}{% Abkuerzung fuer Siehe unten
\abkwarning%
Siehe unten\xspace}

\newcommand*{\oae}{% Abkuerzung fuer o.ae.
o.\,\"a.\xspace} %???

\newcommand*{\oa}{% Abkuerzung fuer oben angegeben
o.\,a.\xspace}

\newcommand*{\og}{% Abkuerzung fuer oben genannten
o.\,g.\xspace}

\newcommand*{\mE}{% Abkuerzung fuer meines Erachtens
m.\,E.\xspace}

\newcommand*{\maW}{% Abkuerzung fuer mit anderen Worten
m.\,a.\,W.\xspace}

\newcommand*{\MaW}{% Abkuerzung fuer Mit anderen Worten
\abkwarning%
Mit anderen Worten\xspace}

\newcommand*{\mwN}{% Abkuerzung fuer m.w.N.
m.\,w.\,N.\xspace}

\newcommand*{\nF}{% Abkuerzung fuer neue Fassung
n.\,F.\xspace}

\newcommand*{\usw}{% Abkuerzung fuer u.s.w.
u.\,s.\,w.\xspace}

\newcommand*{\idR}{% Abkuerzung fuer in der Regel
i.\,d.\,R.\xspace}

\newcommand*{\ua}{% Abkuerzung fuer unter anderem
u.\,a.\xspace}

\newcommand*{\uae}{% Abkuerzung fuer u.ae.
u.\,\"a.\xspace} %???

\newcommand*{\uU}{% Abkuerzung fuer unter Umstaenden
u.\,U.\xspace}

\newcommand*{\vAw}{% Abkuerzung fuer von Amts wegen
v.\,A.\,w.\xspace}

\newcommand*{\va}{% Abkuerzung fuer vor allem
v.\,a.\xspace}

\newcommand*{\zT}{% Abkuerzung fuer zum Teil
z.\,T.\xspace} 

% ========
% Randnummern
\setlength{\marginparwidth}{15mm}
\marginparsep1mm
\newcounter{randnummer}
\normalmarginpar

\newcommand{\rn}[1]{%
\refstepcounter{randnummer}{#1}%
\marginnote{\quad\textbf{\arabic{randnummer}}}}

% ========
% Listen
\newlist{steps}{itemize}{8}
\setlist[steps,1]{label=\textbullet}
\setlist[steps,2]{label=\textasteriskcentered}
\setlist[steps,3]{label=$\diamondsuit$}
\setlist[steps,4]{label=$\star$}
\setlist[steps,5]{label=$\alpha$}
\setlist[steps,6]{label=\Square}
\setlist[steps,7]{label=\fullnote}
\setlist[steps,8]{label=\varhexstar}