% Beginne Intro
\begingroup
% Zeilenabstand innerhalb der itemize- und enumerate-Umgebungen auf Null setzen
\setlist{noitemsep}
%Dokumentenvorspann, also Vorwort, Inhalt usw.
%R"omische Ziffern (gro"s) zur Nummerierung des Vorspanns
\pagenumbering{Roman}

% Daten f"ur Titelseite definieren
% Titel der Arbeit
\title{Script "Offentlichen Recht}

% Ersteller mit Adresse
\author{Sebastian Muster\\Musterstra"se 30\\08151 Musterstadt}

%Titelseite setzen
\begin{titlepage}

%Autor/Adresse setzen, rechts das Datum, dazwischen auff"ullen
\makeatletter
\noindent\@author\hfill\@date
\makeatother

\vspace{3cm}

%% Den ganzen Text auf der Titelseite zentrieren
\begin{center}

{%\sffamily
\bfseries\hrulefill

\makeatletter
\Huge\@title
\makeatother


\hrulefill}

\end{center}
\end{titlepage}

\chapter*{Vorwort}
Der Bundestag beschlie"st ein Gesetz. Der Bundespr"asident findet das nicht so prickelnd und will es nicht
unterschreiben. Er fragt einen Feld-, Wald- und Wiesenanwalt, ob er muss.
\clearpage

\setcounter{tocdepth}{3}
% Inhaltsverzeichnis (Gliederung) erstellen
\tableofcontents

%Literaturverzeichnis jurlit.bib
%Der Seitenrand spielt hier irgendwie nicht mit :(
\nocite{*} % alle Eintrge in der Bib anzeigen
\bibliography{script}
\bibliographystyle{jurabib}
Gebraucht werden die üblichen Abkürzungen, vgl. Kirchner,
Hildebert/Butz, Cornelie: Abkürzungsverzeichnis der Rechtssprache, 5.
Aufl., Berlin/New York 2003

%Stellt sicher, dass die letzte Seite ausgegeben ist, bevor die Gruppe endet.
%Diese L"osung funktioniert m"oglicherweise nicht mit k"unftigen Versionen von KOMA-Script!
%http://groups.google.de/group/de.comp.text.tex/browse_thread/thread/a4cd123873925372/eeebf6ccf4cfac04?hl=de#eeebf6ccf4cfac04
\clearpage

%Ende des Teils mit normalem Seitenrand
\endgroup