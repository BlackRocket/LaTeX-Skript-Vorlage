\chapter{Abkürzungsverzeichnis}
\begin{acronym}[MittRhNotK]
\setlength{\itemsep}{-\parsep} % geringerer Zeilenabstand
% Format der Abkürzungsdefinition: \acro{}[]{}
% {Verweis}[Abkürzung]{ausgeschriebene Abkürzung}
\acro{aA}[a.A.]{anderer Ansicht}
\acro{allg}[allg.]{allgemein}
\acro{aaO}[a.a.O]{am angebenen Ort}
\acro{Abs}[Abs.]{Absatz}
\acro{aE}[a.E.]{am Ende}
\acro{aF}[a.F.]{alte Fassung}
\acro{AG}[AG]{Amtsgericht}
\acro{AGB}[AGB]{Allgemeine Geschäftsbedingungen}
\acro{AGG}[AGG]{Allgemeines Gleichbehandlungsgesetz}
\acro{AktG}[AktG]{Aktiengesetz}
\acro{Anm}[Anm]{Anmerkung}
\acro{ArbG}[ArbG]{Arbeitsgericht}
\acro{ArbGG}[ArbGG]{Arbeitsgerichtsgesetz}
\acro{arge}[arg.e.]{Argument aus}
\acro{Art}[Art.]{Artikel}
\acro{Az}[Az.]{Aktenzeichen}

\acro{BAG}[BAG]{Bundesarbeitsgericht}
\acro{BAGE}[BAGE]{Entscheidungen des Bundesarbeitsgerichts}
\acro{Bd}[Bd.]{Band}
\acro{begr}[Begr.]{Begründer}
\acro{Begr}[Begr.]{Begründung}
\acro{Bde}[Bde.]{Bände}
\acro{BGB}[BGB]{Bürgerliches Gesetzbuch}
\acro{bgbaf}[BGB aF.]{Bürgerliches Gesetzbuch alte Fassung (bis 31.08.2009)}
\acro{Beschl}[Beschl.]{Beschluss}
\acro{Bf}[Bf.]{Beschwerdeführer}
\acro{BGBl}[BGBl]{Bundesgesetzblatt}
\acro{BGH}[BGH]{Bundesgerichtshof}
\acro{BGHZ}[BGHZ]{Entscheidungen in Zivilsachen des Bundesgerichtshofes}
\acro{BTDrs}[BT-Drs.]{Bundestagsdrucksache}
\acro{BTDrucks}[BT-Drucks.]{Bundestags Drucksachen}
\acro{Bg}[Bg.]{Beschwerdegegner}
\acro{BVerfG}[BVerfG]{Bundesverfassungsgericht}
\acro{BVerfGE}[BVerfGE]{Entscheidungen des Bundesverfassungsgerichts}
\acro{BGHSt}[BGHSt]{Entscheidungen in Strafsachen des Bundesgerichtshofes}
\acro{bzgl}[bzgl.]{bezüglich}
\acro{Bspe}[Bsp.(e)]{Beispiel(e)}

\acro{ders}[ders.]{derselbe}
\acro{dh}[d.h.]{das heißt}
\acro{Diss}[Diss.]{Dissertation}
\acro{Def}[Def.]{Definition}

\acro{EG}[EG]{Europäische Gemeinschaft}
\acro{Einl}[Einl.]{Einleitung}
\acro{eV}[e.V.]{eingetragener Verein}
\acro{Erl}[Erl.]{Erlass}

\acro{f}[f.]{folgende,für}
\acro{ff}[ff.]{die folgenden}
\acro{FamRZ}[FamRZ]{Zeitschrift für das gesamte Familienrecht}
\acro{Fn}[Fn.]{Fussnote}
\acro{fpr}[FPR]{Familie, Partnerschaft, Recht (Zeitschrift)}
\acro{FS}[FS]{Festschrift}

\acro{G}[G]{Gesetz}
\acro{GBO}[GBO]{Grundbuchordnung}
\acro{GewO}[GewO]{Gewerbeordnung}
\acro{GG}[GG]{Grundgesetz}
\acro{GmbH}[GmbH]{Gesellschaft mit beschränkter Haftung}
\acro{GVG}[GVG]{Gerichtsverfassungsgesetz}
\acro{ggf}[ggf.]{gegebenenfalls}
\acro{GOA}[GOA]{Geschäftsführung ohne Auftrag}

\acro{Hrsg}[Hrsg.]{Herausgeber}
\acro{HGB}[HGB]{Handelsgesetzbuch}
\acro{hM}[h.M.]{herrschende Meinung}
\acro{hL}[h.L.]{herrschende Lehre}
\acro{Hs}[Hs.]{Halbsatz}

\acro{iA}[i.A.]{im Allgemeinen}
\acro{idR}[i.d.R.]{in der Regel}
\acro{idF}[i.d.F.]{in der Fassung}
\acro{idS}[i.d.S.]{in dem Sinne}
\acro{ieS}[i.e.S.]{im engeren Sinne}
\acro{iU}[i.U.]{im Übrigen}
\acro{iSd}[i.S.d.]{im Sinne des}
\acro{iE}[i.E.]{im Ergebnis}
\acro{iVm}[i.V.m.]{in Verbindung mit}
\acro{iHv}[i.H.v.]{in Höhe von}
\acro{iwS}[i.w.S.]{im weiteren Sinne}
\acro{iR}[i.R.]{im Recht}
\acro{iRd}[i.R.d.]{im Rahmen der}
\acro{iZ}[i.Z.]{im Zweifel}
\acro{insb}[insb.]{insbesondere}

\acro{JA}[JA]{Juristische Arbeitsblätter}
\acro{Jura}[Jura]{Juristische Ausbildung}
\acro{JuS}[JuS]{Juristische Schulung}
\acro{JuSL}[JuSL]{JuS Lernbögen}

\acro{KV}[KV]{Kaufvertrag}
\acro{KG}[KG]{Kammergericht}
\acro{Kj}[Kj]{Kalenderjahr}

\acro{Lit}[Lit.]{Literatur}
\acro{LG}[LG]{Landesgericht}
\acro{lt}[lt.]{laut}

\acro{maW}[m.a.W.]{mit anderen Worten}
\acro{mE}[m.E.]{meines Erachtens}
\acro{MittRhNotK}[MittRhNotK]{Mitteilungen der Rheinischen Notarkammer (Mitteilungsblatt)}
\acro{MM}[M.M.]{Mindermeinung}
\acro{mN}[m.N.]{mit Nachweisen}
\acro{MuKo}[MüKo]{Münchener Kommentar}
\acro{mWv}[m.W.v.]{mit Wirkung von}
\acro{mwN}[m.w.N.]{mit weiteren Nachweisen}

\acro{Nachw}[Nachw.]{Nachweis}
\acro{nF}[n.F.]{neue Fassung}
\acro{njoz}[NJOZ]{Neue Juristische Online Zeitschrift (Zeitschrift)}
\acro{NJW}[NJW]{Neue Juristische Wochenzeitschrift}
\acro{NJWRR}[NJW-RR]{NJW Rechtssprechungs Report}
\acro{Nr}[Nr.]{Nummer}

\acro{oa}[o.a.]{oben angegeben}
\acro{obj}[obj.]{objektiv}
\acro{og}[o.g.]{oben genannten}
\acro{OLG}[OLG]{Oberlandesgericht}
\acro{OHG}[OHG]{Offene Handelsgesellschaft}

\acro{PartG}[PartG]{Parteiengesetz}
\acro{Pet}[Pet.]{Petitionen}

\acro{RA}[RA]{Rechtsanwalt}
\acro{RAe}[RAe]{Rechtsanwälte}
\acro{RAe}[RAe]{Rechtsanwälte}
\acro{Rspr}[Rspr.]{Rechtssprechung}
\acro{RF}[RF]{Rechtsfolge}
\acro{RFen}[RF(en)]{Rechtsfolge(n)}
\acro{Rdnr}[Rdnr.]{Randnummer}
\acro{RG}[RG]{Reichsgericht}
\acro{rn}[Rn.]{Randnummer}

\acro{str}[str.]{streitig}
\acro{s}[s.]{siehe}
\acro{S}[S.]{Satz}
\acro{sa}[s.a.]{siehe auch}
\acro{subj}[subj.]{subjektiv}
\acro{SV}[SV]{Sachverhalt}
\acro{so}[s.o.]{siehe oben}
\acro{sog}[sog.]{sogenannt}
\acro{StGB}[StGB]{Strafgesetzbuch}
\acro{StPO}[StPO]{Strafprozessordnung}
\acro{stRspr}[st.Rspr.]{ständige Rechtssprechung}
\acro{StVO}[StVO]{Strassenverkehrsordnung}
\acro{su}[s.u.]{siehe unten}

\acro{Tb}[Tb.]{Tatbestand}

\acro{u}[u.]{und}
\acro{ua}[u.a.]{unter anderem}
\acro{usw}[u.s.w.]{und so weiter}
\acro{uU}[u.U.]{unter Umständen}

\acro{v}[v.]{vom}
\acro{vAw}[v.A.w.]{von Amts wegen}
\acro{va}[v.a.]{vor allem}
\acro{Vor}[Vor.]{Voraussetzung}
\acro{vgl}[vgl.]{vergleiche}
\acro{VA}[VA]{Verwaltungsakt}
\acro{Vfg}[Vfg.]{Verfügung}

\acro{WE}[WE]{Willenserklärung}
\acro{WP}[WP]{Wahlperiode}

\acro{zB}[z.B.]{zum Beispiel}
\acro{ZPO}[ZPO]{Zivilprozessordnung}
\acro{zA}[z.A.]{zur Anstellung}
\end{acronym}
% Format der Abkürzungsdefinition: \acro{}[]{}


