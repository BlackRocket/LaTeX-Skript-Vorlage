% ========
% Diverse Boxen
% Inspiriert durch die mdframed Beispiele und jur-info.sty von Peter Felix Schuster (http://www.peterfelixschuster.de/)

% Fieldset Box 
% Verwendung \begin{fieldset}{<TITEL>}<INHALT>\end{fieldset}
\tikzstyle{titregris} =
     [draw=gray, thick, fill=white, shading = exersicetitle, %
      text=gray, rectangle, rounded corners, right,minimum height=.7cm]
\pgfdeclarehorizontalshading{exersicebackground}{100bp}
          {color(0bp)=(green!40); color(100bp)=(black!5)}
\pgfdeclarehorizontalshading{exersicetitle}{100bp}
          {color(0bp)=(red!40);color(100bp)=(black!5)}

\makeatletter
\def\mdf@@theexercisetitle{}%new mdframed key:
\define@key{mdf}{theexercisetitle}{%
    \def\mdf@@theexercisetitle{#1}
}
\mdfdefinestyle{exercisestyle}{%
  outerlinewidth=1em,outerlinecolor=white,%
  leftmargin=-1em,rightmargin=-1em,%
  middlelinewidth=1.2pt,roundcorner=5pt,linecolor=gray,
  apptotikzsetting={\tikzset{mdfbackground/.append style ={%
                       shading = exersicebackground}
                       }},
  innertopmargin=1.2\baselineskip,
  skipabove={\dimexpr0.5\baselineskip+\topskip\relax},
  skipbelow={-1em},
  needspace=3\baselineskip,
  frametitlefont=\sffamily\bfseries,
    %frametitlebackgroundcolor=exersicetitle,
  settings={\global},
   singleextra={%
        \node[titregris,xshift=1cm] at (P-|O) %
           {~\mdf@frametitlefont{\mdf@@theexercisetitle}~};
     },
    firstextra={%
        \node[titregris,xshift=1cm] at (P-|O) %
           {~\mdf@frametitlefont{\mdf@@theexercisetitle}~};
     },
}
\makeatother

\newenvironment{fieldset}[1]% 
   {\begin{mdframed}[style=exercisestyle,theexercisetitle={#1}]}% 
   {\end{mdframed}}% 
\SetBlockEnvironment{fieldset}

% Vor dem Text ein Stopschild und roter Text 
% Verwendung \begin{stopschild} <INHALT> \end{stopschild}
\tikzset{
  warningsymbol/.style={
      rectangle,
      fill=white,scale=1,
      overlay}}
\mdfdefinestyle{warning}{%
 hidealllines=true,leftline=true,
 skipabove=12,skipbelow=12pt,
 innertopmargin=0.4em,%
 innerbottommargin=0.4em,%
 innerrightmargin=0.7em,%
 rightmargin=0.7em,%
 innerleftmargin=1.7em,%
 leftmargin=0.7em,%
 middlelinewidth=.2em,%
 linecolor=red,%
 fontcolor=red,%
 firstextra={\path let \p1=(P), \p2=(O) in ($(\x2,0)+0.5*(0,\y1)$)
                           node[warningsymbol] {{\huge \Stopsign}};},%
 secondextra={\path let \p1=(P), \p2=(O) in ($(\x2,0)+0.5*(0,\y1)$)
                           node[warningsymbol] {{\huge \Stopsign}};},%
 middleextra={\path let \p1=(P), \p2=(O) in ($(\x2,0)+0.5*(0,\y1)$)
                           node[warningsymbol] {{\huge \Stopsign}};},%
 singleextra={\path let \p1=(P), \p2=(O) in ($(\x2,0)+0.5*(0,\y1)$)
                           node[warningsymbol] {{\huge \Stopsign}};},%
}
\newenvironment{stopschild}% 
   {\begin{mdframed}[style=warning]}% 
   {\end{mdframed}}% 
\SetBlockEnvironment{stopschild}

% Verweis Box 
% Verwendung \begin{verweis}{<TITEL>}<INHALT>\end{verweis}
\tikzset{
   excursus arrow/.style={%
      line width=2pt,
      draw=gray!40,
      rounded corners=2ex,
   },
   excursus head/.style={
      fill=white,
      font=\bfseries\sffamily,
      text=gray!80,
      anchor=base west,
   },
}
\makeatletter
\def\mdf@@thetitle{}%new mdframed key:
\define@key{mdf}{thetitle}{%
    \def\mdf@@thetitle{#1}
}
\mdfdefinestyle{digressionarrows}{%
 singleextra={%
      \path let \p1=(P), \p2=(O) in (\x2,\y1) coordinate (Q);
      \path let \p1=(Q), \p2=(O) in (\x1,{(\y1-\y2)/2}) coordinate (M);
      \path [excursus arrow, round cap-to]
         ($(O)+(5em,0ex)$) -| (M) |- %
         ($(Q)+(12em,0ex)$) .. controls +(0:16em) and +(185:6em) .. %
         ++(23em,2ex);
      \node [excursus head] at ($(Q)+(2.5em,-0.75pt)$) {\mdf@@thetitle};},
 firstextra={%
      \path let \p1=(P), \p2=(O) in (\x2,\y1) coordinate (Q);
      \path [excursus arrow,-to]
         (O) |- %
         ($(Q)+(12em,0ex)$) .. controls +(0:16em) and +(185:6em) .. %
         ++(23em,2ex);
      \node [excursus head] at ($(Q)+(2.5em,-2pt)$) {\mdf@@thetitle};},
 secondextra={%
      \path let \p1=(P), \p2=(O) in (\x2,\y1) coordinate (Q);
      \path [excursus arrow,round cap-]
         ($(O)+(5em,0ex)$) -| (Q);},
 middleextra={%
      \path let \p1=(P), \p2=(O) in (\x2,\y1) coordinate (Q);
      \path [excursus arrow]
         (O) -- (Q);},
   middlelinewidth=2.5em,middlelinecolor=white,
   hidealllines=true,topline=true,
   innertopmargin=0.5ex,
   innerbottommargin=2.5ex,
   innerrightmargin=2pt,
   innerleftmargin=2ex,
   skipabove=0.87\baselineskip,
   skipbelow=0.62\baselineskip,
}
\makeatother
\newenvironment{verweis}[1]% 
   {\begin{mdframed}[style=digressionarrows,thetitle={#1}]}% 
   {\end{mdframed}}% 
\SetBlockEnvironment{verweis}

% Graue Box 
% Verwendung \begin{greybox} <INHALT> \end{greybox}
\global\mdfdefinestyle{greybox}{%
     outerlinewidth=0.1mm,innerlinewidth=0pt,
     linecolor=black,roundcorner=5pt,backgroundcolor=grey97,
     innerleftmargin=5pt;innerrightmargin=5pt,innertopmargin=5pt.innerbottommargin=5pt,
}
\newenvironment{greybox}% 
   {\begin{mdframed}[style=greybox]}% 
   {\end{mdframed}}% 
\SetBlockEnvironment{greybox}

% Weiße Box 
% Verwendung \begin{whitebox} <INHALT> \end{whitebox}
\global\mdfdefinestyle{whitebox}{%
     outerlinewidth=0.1mm,innerlinewidth=0pt,
     linecolor=black,roundcorner=5pt,backgroundcolor=white,
     innerleftmargin=5pt;innerrightmargin=5pt,innertopmargin=5pt.innerbottommargin=5pt,
}
\newenvironment{whitebox}% 
   {\begin{mdframed}[style=whitebox]}% 
   {\end{mdframed}}% 
\SetBlockEnvironment{whitebox}

% Blaue Info Box 
% Verwendung \begin{infobox} <INHALT> \end{infobox}
\newcommand{\infosymbol}[1]{\noindent\umfluss[2]{\Huge #1}}

\global\mdfdefinestyle{infobox}{%
     outerlinewidth=0.1mm,innerlinewidth=0pt,
     linecolor=blau2,roundcorner=5pt,backgroundcolor=blau1,
     innerleftmargin=5pt;innerrightmargin=5pt,innertopmargin=5pt.innerbottommargin=5pt,
}
\newenvironment{infobox}% 
   {\begin{mdframed}[style=infobox, frametitle={{\LARGE \Info \hspace{0.3em}}{\Large Hinweis} \vspace{0.5em}}]}% 
   {\end{mdframed}}% 
\SetBlockEnvironment{infobox}

% WICHTIG Box 
% Verwendung \begin{wichtibox} <INHALT> \end{wichtigbox}
\global\mdfdefinestyle{wichtigbox}{%
     outerlinewidth=0.2mm,innerlinewidth=0pt,
     linecolor=red,roundcorner=5pt,backgroundcolor=white,fontcolor=black,
     innerleftmargin=5pt;innerrightmargin=5pt,innertopmargin=5pt.innerbottommargin=5pt,
}
\newenvironment{wichtigbox}% 
   {\begin{mdframed}[style=wichtigbox,frametitle={\textcolor{red}{{\Large \Lightning \hspace{0.2em} Wichtig}} \vspace{0.5em}}] }% 
   {\end{mdframed}}% 
\SetBlockEnvironment{wichtigbox}

% Schemata Box 
% Verwendung \begin{schema}{<TITEL>} <INHALT> \end{schema}
\mdtheorem[%
 apptotikzsetting={\tikzset{mdfbackground/.append style =%
                              {top color=blue!20!white,
                               bottom color=blue!20!white},
                            mdfframetitlebackground/.append style =%
                               {top color=blue!10!white,
                                bottom color=blue!10!white}
                           }%
                    },
  ,roundcorner=10pt,middlelinewidth=0pt,
  shadow=false,frametitlerule=true,frametitlerulewidth=1pt,
  innertopmargin=10pt,%
  ]{alternativschema}{Schema}
  
\newenvironment{schema}[1]% 
   {\begin{alternativschema}[{#1}]}% 
   {\end{alternativschema}}% 
\SetBlockEnvironment{schema}
