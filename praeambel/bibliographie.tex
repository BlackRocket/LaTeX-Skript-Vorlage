% ========
% Bibliographie und Fußnoten

% Konfiguriert das jurabib-Paket. Kommentare aus der Dokumentation
\jurabibsetup{%
	authorformat=italic, 		% Autor kursiv
	titleformat={commasep,all},	% Komma zwischen Autor/Bearbeiter und Titel im Zitat;
	annotatorformat=italic, 	% Bearbeiter kursiv
	annotatorlastsep=divis, 	% Bearbeiter nach Bindestrich
	commabeforerest, 			% Komma nach Verfasser (vor dem Rest)
	crossref={long,dynamic}, 	% Lange Querverweise (auf Festschriften etwa)
	howcited=compare, 			%"zitiert als...", wenn shorttitle anders als title
	pages={always,test}, 		% zitierten Seitenbereich immer ausgeben (always),
	bibformat={tabular,ibidem},	% Litverz. tabellarisch, mit der-/dieselbe
	lookforgender, 				% Auf das gender-Feld achten, um ders./dies. Zitate zu ermglichen
	superscriptedition=switch, 	% Hochgestellte Auflage, wenn ssedition=1 in .bib
	dotafter=bibentry, 			% Punkt nach jedem Eintrag im Lit.verzeichnis
}

\citetitlefortype{article,periodical,incollection} % Diese immer mit Titel zitieren
\formatpages[~]{article}{(}{)} % Zeitschriften als JZ 2001, 1057, (S.) %[, ]
\formatpages[~]{incollection}{(}{)} % Sammbelbandbeiträge als FS xy, 1057, (S.) %[, ]

%Bei Festschriften und Zeitschriftenartikeln: "`in"' vor Titel der Sammlung
\renewcommand{\bibjtsep}{In: } 
\renewcommand{\bibbtsep}{In: } 

%Bei Periodika (AcP et.al.) die Jahreszahl in runde (statt eckige) Klammern setzen.
\renewcommand*{\bibpldelim}{(}
\renewcommand*{\bibprdelim}{)}

%Linke Spalte des Lit.verz. soll ein Drittel der ges. Textbreite einnehmen
\renewcommand*{\bibleftcolumn}{\textwidth/3}

%Nicht Punkt, sondern Komma nach Auflage
\DeclareRobustCommand{\jbaensep}{,}

%Bei Artikeln: Heft-Nummer in Klammern hinter dem Erscheinungsjahr, etwa 2002(7). (aus jurabib-Gruppe)
\DeclareRobustCommand{\artnumberformat}[1]{(#1)}

%Kein Komma hinter Zeitschriftenname (aus: jurabib-Gruppe #661)
\AddTo\bibsgerman{\def\ajtsep{}}

% Keine hochgestellten Ziffern in der Fussnote (KOMA-Script-spezifisch):
\deffootnote{1.5em}{1em}{\makebox[1.5em][l]{\thefootnotemark}}

% Abstand Text <-> Fussnote
\addtolength{\skip\footins}{\baselineskip}

% Verhindert das Fortsetzen von Fussnoten auf der gegenüberligenden Seite
\interfootnotelinepenalty=10000