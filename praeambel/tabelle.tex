% neuer Tabellenspaltentyp D (X zentriert)
\newcolumntype{D}{>{\centering\arraybackslash}X}

%**************************************************************
% Makro für Tabellen aus dem Quellcode des freiesMagazin (http://www.freiesmagazin.de/)
% Copyright: 2006-2010 Eva Drud, Dominik Wagenfuehr, Thorsten Panknin, Dominik Honnef
% Lizenz: CC-BY-SA 3.0 Unported http://creativecommons.org/licenses/by-sa/3.0/deed.de
%**************************************************************
% neuer Spaltentyp für Tabellen
\newcolumntype{C}[1]{>{\centering}p{#1}}

\newcommand{\PDFonly}[1]{#1}
% Vor einem Abstand die Zeile strecken.
\newboolean{AbstandSoftLineBreak}

% Vor einem Abstand die Zeile nicht hart umbrechen.
\newboolean{AbstandHardLineBreak}

% Nach einem Abstand die Spalte beenden.
\newboolean{AbstandColumnBreak}

% Setzen der Abstand-Standardwerte
\newcommand*{\SetDefaultAbstandKeys}
{%
    \setboolean{AbstandSoftLineBreak}{false}%
    \setboolean{AbstandHardLineBreak}{false}%
    \setboolean{AbstandColumnBreak}{false}%
}

% Abstand im Text, der einen kompletten Bereich frei lässt.
% Benutzung: \Abstand[OPTIONEN]{HOEHE}
% Beispiel: \Abstand{0.7em} oder \Abstand[stretch]{7.1em}
% Optionen:
%   stretch - Fügt vor dem Abstand einen sanften Zeilenumbruch ein,
%             sodass eine Zeile bis zum Spaltenende gestreckt wird.
%   linebreak - Fügt einen harten Zeilenumbruch ein.
%   break - Beendet nach dem Abstand die Spalte.
\newcommand*{\Abstand}[2][]
{%
    \PDFonly{%
        \SetDefaultAbstandKeys{}%
        \setkeys{Abstand}{#1}%
        \ifthenelse{\boolean{AbstandSoftLineBreak}}{\sbreak}{}%
        \ifthenelse{\boolean{AbstandHardLineBreak}}{\\}{}%
        \vspace*{#2}%
        \ifthenelse{\boolean{AbstandColumnBreak}}{\cbr

        }{\par{}}%
    }%
}

% Abstand hinter einer Tabelle
\newlength{\TabellenVSkip}
% Setzen der Tabellen-Standardwerte
\newcommand*{\SetDefaultTabellenKeys}[1]
{%
    \setlength{\TabellenVSkip}{0em}%
}

% Makro für Dateien, Ordner, Verzeichnisse, Befehle, Optionen etc. im Terminal
\newcommand*{\term}[1]{\textbf{\texttt{#1}}}
\newcommand*{\termZ}[2]{\textbf{\texttt{#1}} \textbf{\texttt{#2}}}
\newcommand*{\termD}[3]{\textbf{\texttt{#1}} \textbf{\texttt{#2}} \textbf{\texttt{#3}}}

% Makro für Tabellen aus dem Quellcode des freiesMagazin (http://www.freiesmagazin.de/)
% Benutzung: \begin[OPTIONEN]{Tabelle}{SPALTENANZAHL}{SPALTENDEFINITION}{UEBERSCHRIFT} ... \end{Tabelle}
% Beispiel: \begin[OPTIONEN]{Tabelle}{3}{|p{1cm}cc|}{Tabelle 1} erzeugt eine dreispaltige Tabelle, wobei die erste Spalte 1 cm und linksbündig ist, die anderen beiden zentriert

\makeatletter
% Abstand hinter einer Tabelle
\define@key{Tabelle}{vskip}{\setlength{\TabellenVSkip}{#1}}
\makeatother
\newenvironment{Tabelle}[4][]
{%
    \SetDefaultTabellenKeys{}%
    \setkeys{Tabelle}{#1}%
    \begin{minipage}{\linewidth}%
        \centering{}%
        \begin{footnotesize}%
            \renewcommand{\arraystretch}{1.2}% erhöht den Zeilenabstand für Ueberschrift
            \setlength{\arrayrulewidth}{2pt}%
            \rowcolors{1}{mittelgrau}{white}% alternierende Farben
            \arrayrulecolor{darkgrey}%
            \begin{tabular}{#3}
                \firsthline
                \multicolumn{#2}{|>{\columncolor{darkgrey}}c|}%
                {\normalsize \textcolor{white}{\textbf{#4}}}\\
}
{%
                \lasthline
            \end{tabular}%
            \rowcolors{1}{white}{white}%
            \renewcommand{\arraystretch}{1}%
        \end{footnotesize}%
    \end{minipage}%
    \Abstand{\TabellenVSkip}
}