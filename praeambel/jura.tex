% ========
% Jura

%**************************************************************
% Entnommen jur-rdnr.sty
% Peter Felix Schuster (http://www.peterfelixschuster.de)
%**************************************************************
\makeatletter
% Zähler für Fälle und Abwandlungen definieren und setzen
\newcommand{\@rdnrname}{Rn.\,} % Makro für die Bezeichnung für Fälle, macht es flexibel zu ändern
\newcounter{randnr} % Zähler randnr (für Randnummer)
\newcounter{zwrandnr}[randnr] % Zähler zwrandnr (für Zwischen-Randnummer), eine Erhöhung von randnr setzt ihn zurück
\setcounter{randnr}{0} % Anfangs auf 0 setzen
\setcounter{zwrandnr}{0} % Anfangs auf 0 setzen
\newcommand{\newrandnummer}{\refstepcounter{randnr}} % Makro, das den Zähler erhöht
\newcommand{\newzwischenrandnummer}{\refstepcounter{zwrandnr}} % Makro, das den Zähler erhöht
\renewcommand{\therandnr}{\arabic{randnr} }% In arabischen Ziffern nummerieren.
\renewcommand{\thezwrandnr}{\therandnr\,\alph{zwrandnr}} % Kleinbuchstaben nach ''normaler'' Randnummer.
\renewcommand{\p@randnr}{\@rdnrname} % Bei Verweisen Rdnr. vor Zahl ausgegeben
\renewcommand{\p@zwrandnr}{\@rdnrname} % Bei Verweisen Rdnr. vor Zahl\,Buchstabe ausgegeben

% marginnote Config
\setlength{\marginparwidth}{10mm}
\marginparsep2mm
\normalmarginpar

% Zähler über Makro erhöhen und am Rand als Randnummer ausgeben
\newcommand{\randnummer}{\newrandnummer\marginnote{\strut\\\quad\textbf{\therandnr}}}

% Makro für Zwischenrandnummern
\newcommand{\zwischenrandnummer}{\newzwischenrandnummer\marginnote{\strut\\\quad\textbf{\thezwrandnr}}}
\makeatother
%**************************************************************

% Formatierung für nicht nummerierte Listen 
\makeatletter
\renewcommand{\labelitemi}{$\m@th\bullet$} 
\renewcommand{\labelitemii}{--}
\renewcommand{\labelitemiii}{$\m@th\circ$}
\renewcommand{\labelitemiv}{$\m@th\rightarrow$}
\makeatother

% Formatierung für nummerierte Listen [I, 2, c) dd)]
\renewcommand{\labelenumi}{\Roman{enumi}.}
\renewcommand{\labelenumii}{\arabic{enumii}.}
\renewcommand{\labelenumiii}{\alph{enumiii})}
\renewcommand{\labelenumiv}{\alph{enumiv}\alph{enumiv})}



%Abkürzungen - vornehmlich für schmale Zwischenräume (\,) nützlich
%benötigt für einige Befehle jurabib. Befehl nachschlagen, mit dem überprüft werden kann, ob geladen!
\newcommand{\Abs}[1]{\acs{Abs}\,#1\xspace}
\newcommand{\aF}{\acs{aF}\xspace}
\newcommand{\aM}{\acs{aM}\xspace}
\newcommand{\Art}{\acs{Art}\xspace}
\newcommand{\euro}[1]{#1\,\EUR\xspace}
\newcommand{\f}{\acs{f}\xspace}
\newcommand{\ff}{\acs{ff}\xspace}
\newcommand{\hL}{\acs{hL}\xspace}
\newcommand{\hM}{\acs{hM}\xspace}
\newcommand{\idR}{\acs{idR}\xspace}
\newcommand{\iHv}{\acs{iHv}\xspace}
\newcommand{\iSd}{\acs{iSd}\xspace} %%
\newcommand{\iSv}{\acs{iSv}\xspace}
\newcommand{\iVm}{\acs{iVm}\xspace}
\newcommand{\ja}{\ding{'63}\xspace}
\newcommand{\mwN}{\acs{mwN}\xspace}
\newcommand{\LWL}{{\bfseries\underline{\emph{\acs{LWL}}}}} % Literaturwunschliste, also für Blindzitate.
\newcommand{\nein}{\ding{'67}\xspace}
\newcommand{\nF}{\acs{nF}\xspace} 
\newcommand{\Nr}[1]{\acs{Nr}\,#1\xspace}
\newcommand{\pg}[1]{\S\,#1} %\pg{x} => (Paragraf) x
\newcommand{\Pg}[1]{\SSS\,#1} %\Pg{x} => (Paragrafen) x 
\newcommand{\pgAbs}[2]{\S\,#1 Abs.\,#2} %\pgAbs{x}{y} => (Paragraf) x Abs. y
\newcommand{\pgAbsS}[3]{\S\,#1 Abs.\,#2 S.\,#3}
\newcommand{\pgRn}[2]{\S\,#1 Rn.\,#2} %\pg{x}{y} => (Paragraf) x Rn. y
\newcommand{\pgS}[2]{\S\,#1 S.\,#2}
\newcommand{\Rn}[1]{Rn.\,#1}
\newcommand{\zB}{\acs{zB}\xspace}
\newcommand{\fremdwort}[2]{\foreignlanguage{#1}{\itshape{}#2}} %\fremdwort{#1=Sprache}{#2=Text}
\newcommand{\qll}[1]{\emph{#1}}%           fuer Quellen
\newcommand{\code}[1]{\texttt{#1}}%        fuer Computeranweisungen, tags o.ae.
\newcommand{\marke}[1]{{\scshape #1}}%     Markennamen % \texttrademark (TM) oder \textregistered (R)?
\newcommand{\firma}[1]{{\scshape #1}}%     Unternehmensbezeichnung
\newcommand{\prdbez}[1]{{\scshape #1}\index{#1}} %Produktbezeichnung

\newcommand*{\arr}{\(\rightarrow\)\space}
\newcommand*{\arrr}{\(\longrightarrow\)\space}
\newcommand*{\Arr}{\(\Rightarrow\)\space}
\newcommand*{\Arrr}{\(\Longrightarrow\)\space}
\newcommand*{\lrarr}{\(\leftrightarrow\)\space}
\newcommand*{\larr}{\(\leftarrow\)}
\newcommand*{\Larr}{\(\Leftarrow\)}
\newcommand*{\darr}{\(\downarrow\)}
\newcommand*{\Darr}{\(\Downarrow\)}
\newcommand*{\carr}{\mbox{$\curvearrowright$}}

\newcommand{\glink}[4]{\href{http://dejure.org/gesetze/#4/#2.html}{#1\xspace#2\xspace#3\xspace#4}}
